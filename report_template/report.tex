\documentclass[a4paper, 11pt]{article}
\usepackage{graphicx}
\usepackage{amsmath}
\usepackage[pdftex]{hyperref}

% Lengths and indenting
\setlength{\textwidth}{16.5cm}
\setlength{\marginparwidth}{1.5cm}
\setlength{\parindent}{0cm}
\setlength{\parskip}{0.15cm}
\setlength{\textheight}{22cm}
\setlength{\oddsidemargin}{0cm}
\setlength{\evensidemargin}{\oddsidemargin}
\setlength{\topmargin}{0cm}
\setlength{\headheight}{0cm}
\setlength{\headsep}{0cm}

\renewcommand{\familydefault}{\sfdefault}

\title{Introduction to Learning and Intelligent Systems - Spring 2015}
\author{taubnert@student.ethz.ch\\ junkerp@student.ethz.ch\\ kellersu@student.ethz.ch\\}
\date{\today}

\begin{document}
\maketitle

\section*{Project 3 : Image Classification -- Team ``awesome''}

\subsection{Approaches}
We used three different approaches to get the best possible solution for this project: Julia with Mocha library
(neural network), Matlab (neural network), Python (scikit learn)



\subsection{Julia - Mocha.jl}
Mocha is a very well documented nn-library for the programming language Julia. We achieved the best results with not
too many layers and by including a DropoutLayer, which only connects nodes with a certain probability.

\end{document}
